\documentclass[11pt, a4paper]{article}
\usepackage{subfiles}

\input{macro/packages.tex}
\input{macro/settings.tex}
\input{macro/new_commands.tex}


\begin{document}

\author{Mario Rossi\\123456  \and Mario Rossi\\123456 \and Mario Rossi\\123456 \and Mario Rossi\\123456}
\title{\textbf{Management and Analysis of Physics Dataset (mod. A): \\ Hardware accelerated FIR filter and application to an audio stream}}
\maketitle

\section{Aim}
In this project shows and implementation of a FIR filter in FPGA, two different architecture were used. In particular, it's used the Digilent stereo audio PMOD with I$^2$S protocol for communication with the FPGA board and a ADC/DAC ICs. The modules produced were tested in simulations with Python generated input and output along the hardware validation with real word audio samples.





\section{Implementation}
The modules used are listed below and the block diagram shows the various connections.
\begin{itemize}
    \item I$^2$S/AXIS interface;
    \item AXIS FIFO;
    \item FIR filter;
    \item AXIS volume controller;
    
\end{itemize}

Fig.\ref{fig:Block_diagram}.

\vspace{1cm}
\begin{figure}[h!]
    \centering
    \includegraphics[width=0.8\textwidth]{images/BD_gr_cfg.png}
    \caption{\label{fig:Block_diagram} Diagram of top VHDL file.}
\end{figure}
\vspace{1cm}

Now, we describe in details the structure of the main components..

\subsection{I$^2$S to AXIS interface}
This module will take as input the serial data with the I$^2$S protocol and it translates it to an AXI-Stream with the tlast flag used as selector between left and right channel.  

This module will also encode the AXI-Stream to I$^2$S protocol.  

More on this module can be found on the Digilent website \cite{Digilent}. 

\subsection{FIR filter}
We implement a finite impulse response (FIR) filter, which is a filter whose impulse response is of finite duration, this module is AXI-Stream compliant and input and output FIFO are added.
%%%
Firstly, we provide a brief mathematical introduction. Given a sequence $\{x_i\}_{i=1,\dots,N}$ of $N$ input data samples, the output sequence of the filter is obtained by applying the following operation:
\begin{equation}
    \begin{aligned}
        y[n] &=b_{0} x[n]+b_{1} x[n-1]+\cdots+b_{k-1} x[n-k+1] \\
        &=\sum_{i=0}^{k-1} b_{i} \cdot x[n-i]
    \end{aligned}
    \label{eq:FIR}
\end{equation}
which is a convolution operation, or more simply, a weighted moving average. The $b_i$ in Eq. \ref{eq:FIR} are the coefficients that characterize the filter and its order. So, a $k$-th order filter is a filter that works with $k$ coefficients.  

%For our aims
In our work, we consider a 7-th order FIR filter. The values of the coefficients are computed through an online calculator %put ref here
, by setting a cutoff frequency of $4.8$ $kHz$ and a sample rate of $48$ $kHz$. The frequency analysis for this filter setup is showed in Figure \ref{fig:FIR_freq_analysis}.


The values of the coefficients are (16 bits signed integer format):
\begin{align*}
    b_0 &= 1915  \\
    b_1 &= 5389  \\
    b_2 &= 8266  \\
    b_3 &= 9979  \\
    b_4 &= 8266  \\
    b_5 &= 5389  \\
    b_6 &= 1915
\end{align*}

\subsubsection{Latency architecture}
This implementation is mainly focussed to reduce the latency of the filter, to do so a pipelined data flow is employed.
%%%
Firstly let's consider the shift register that will store the last seven inputs of the data stream, this is implemented using a cascade of D flip-flops.
\begin{lstlisting}[style={VHDL-style}]
shift_reg_p : process (clk) is
    begin
        if rising_edge(clk) then
            if (s_new_word = '1') then
                if (s_select = 1) then  -- right audio data 
                    audio_data_shift_r(0) <= signed(s_axis_tdata(AUDIO_DATA_WIDTH-1 downto 0));
                    audio_data_shift_r(1) <= audio_data_shift_r(0);
                    audio_data_shift_r(2) <= audio_data_shift_r(1);
                    audio_data_shift_r(3) <= audio_data_shift_r(2);
                    audio_data_shift_r(4) <= audio_data_shift_r(3);
                    audio_data_shift_r(5) <= audio_data_shift_r(4);
                    audio_data_shift_r(6) <= audio_data_shift_r(5);
                else    -- left audio data
                    audio_data_shift_l(0) <= signed(s_axis_tdata(AUDIO_DATA_WIDTH-1 downto 0));
                    audio_data_shift_l(1) <= audio_data_shift_l(0);
                    audio_data_shift_l(2) <= audio_data_shift_l(1);
                    audio_data_shift_l(3) <= audio_data_shift_l(2);
                    audio_data_shift_l(4) <= audio_data_shift_l(3);
                    audio_data_shift_l(5) <= audio_data_shift_l(4);
                    audio_data_shift_l(6) <= audio_data_shift_l(5);
                end if;
            end if;
        end if;

    end process shift_reg_p;
\end{lstlisting}

When the data are stored the multiplication can take place, here the whole 7 samples are processed, and finally on the following clock cycle the addition is performed.

\begin{lstlisting}[style={VHDL-style}]
process (clk) is
begin
   if rising_edge(clk) then
       if (s_new_packet_r(0) = '1') then  -- multiplication
            
           mult_reg_l(0) <= audio_data_shift_l(0) * to_signed(coeff(0), 16);
           mult_reg_l(1) <= audio_data_shift_l(1) * to_signed(coeff(1), 16);
           mult_reg_l(2) <= audio_data_shift_l(2) * to_signed(coeff(2), 16);
           mult_reg_l(3) <= audio_data_shift_l(3) * to_signed(coeff(3), 16);
           mult_reg_l(4) <= audio_data_shift_l(4) * to_signed(coeff(4), 16);
           mult_reg_l(5) <= audio_data_shift_l(5) * to_signed(coeff(5), 16);
           mult_reg_l(6) <= audio_data_shift_l(6) * to_signed(coeff(6), 16);

           mult_reg_r(0) <= audio_data_shift_r(0) * to_signed(coeff(0), 16);
           mult_reg_r(1) <= audio_data_shift_r(1) * to_signed(coeff(1), 16);
           mult_reg_r(2) <= audio_data_shift_r(2) * to_signed(coeff(2), 16);
           mult_reg_r(3) <= audio_data_shift_r(3) * to_signed(coeff(3), 16);
           mult_reg_r(4) <= audio_data_shift_r(4) * to_signed(coeff(4), 16);
           mult_reg_r(5) <= audio_data_shift_r(5) * to_signed(coeff(5), 16);
           mult_reg_r(6) <= audio_data_shift_r(6) * to_signed(coeff(6), 16);

        elsif (s_new_packet_r(1) = '1') then  -- addition
            data(0) <= mult_reg_l(0) + mult_reg_l(1) + mult_reg_l(2)
 + mult_reg_l(3) + mult_reg_l(4) + mult_reg_l(5) + mult_reg_l(6);
            data(1) <= mult_reg_r(0) + mult_reg_r(1) + mult_reg_r(2)
 + mult_reg_r(3) + mult_reg_r(4) + mult_reg_r(5) + mult_reg_r(6);
        end if;
    end if;
end process;
\end{lstlisting}

From the code above it is clear that in two clock cycles the data are processed, but the price in resources is very high, in this particular case 14 DSP blocks are used (the ARTY A/ 35T has 90 DSPs).

%% table with resources utilization

\subsubsection{Multiplication and ACcumulation (MAC) architecture}

In this acrhitecture the principle is the same, but the accumulation in split in seven different steps. In our case it is used a FSM that will muliply and accumulate all the seven steps.  
in this way only 2 DPSs are needed at the cost of increased latency, in partucular 7 clock cyles per data.

\begin{lstlisting}[style={VHDL-style}]
MAC_p : process (clk, rst) is
begin
    if (rst = '1') then
        state <= idle;
        s_axis_tready_r <= '0';
        m_axis_tvalid_r <= '0';
        res_l   <=  (others => '0') ;
        res_r   <=  (others => '0') ;
        sel     <= '0';
    elsif rising_edge(clk) then
        case state is
            when idle =>
                s_axis_tready_r <= '1';
                m_axis_tvalid_r <= '0';
                res_l   <=  (others => '0') ;
                res_r   <=  (others => '0') ;
                if (s_axis_tvalid = '1') then
                    sel <= s_axis_tlast;
                    state <= mult_0;
                end if;
            when mult_0 =>
                s_axis_tready_r <= '0';
                m_axis_tvalid_r <= '0';
                if (sel = '1') then
                    res_r <= res_r + (audio_data_shift_r(0) * to_signed(coeff(0), 16));
                else
                    res_l <= res_l + (audio_data_shift_l(0) * to_signed(coeff(0), 16));
                end if;
                state <= mult_1;
                .
                .
                .
            when mult_6 =>
                s_axis_tready_r <= '0';
                if (sel = '1') then
                    res_r <= res_r + (audio_data_shift_r(6) * to_signed(coeff(6), 16));
                else
                    res_l <= res_l + (audio_data_shift_l(6) * to_signed(coeff(6), 16));
                end if;
                if (m_axis_tready = '1') then
                    m_axis_tvalid_r <= '1';
                    state <= idle;
                else
                    m_axis_tvalid_r <= '0';
                    state <= send_data;
                end if;
                state <= idle;
            when send_data =>
                s_axis_tready_r <= '0';
                if (m_axis_tready = '1') then
                    m_axis_tvalid_r <= '1';
                    state <= idle;
                else
                    m_axis_tvalid_r <= '0';
                    state <= send_data;
                end if;
        end case;
    end if;
end process MAC_p;
\end{lstlisting}



\section{Module validation}

\subsection{Testbench validation}

The produced modules were validate via simulation. In the testbench the input array is read from file and sent to the FIR filter, later the result are first compared with a file result and then write back to another file.  

A snippet of the code is given below.

\begin{lstlisting}[style={VHDL-style}]
check_data_p : process (clk) is
    --------------------------------------------

    file test_vector                : text open write_mode is "output_file_fir.txt";
    variable row                    : line;
    
    --------------------------------------------
begin

    if(rising_edge(clk)) then
        if (m_axis_tvalid = '1' and m_axis_tlast = '0') then
            value1_fir_24_bit_out <= m_axis_tdata(23 downto 0);
            if (signed(m_axis_tdata(23 downto 0)) < signed(value1_down_out))then
                report "Left output does not match, expected " & integer'image(to_integer(signed(value1_std_logic_24_bit_out))) 
                & " got " & integer'image(to_integer(signed(m_axis_tdata(23 downto 0)))) severity warning;
                err_cnt <= err_cnt + X"0001";
            elsif (signed(m_axis_tdata(23 downto 0)) > signed(value1_up_out)) then
                report "Left output does not match, expected " & integer'image(to_integer(signed(value1_std_logic_24_bit_out))) 
                & " got " & integer'image(to_integer(signed(m_axis_tdata(23 downto 0)))) severity warning;
                err_cnt <= err_cnt + X"0001";
            end if;
        elsif (m_axis_tvalid = '1' and m_axis_tlast = '1') then
            write(row, to_integer(signed(value1_fir_24_bit_out))    , right, 15);
            write(row, to_integer(signed(m_axis_tdata(23 downto 0))), right, 15);
            writeline(test_vector,row);
            value2_fir_24_bit_out <= m_axis_tdata(23 downto 0);
            if (signed(m_axis_tdata(23 downto 0)) < signed(value2_down_out))then
                report "Right output does not match, expected " & integer'image(to_integer(signed(value2_std_logic_24_bit_out))) 
                & " got " & integer'image(to_integer(signed(m_axis_tdata(23 downto 0)))) severity warning;
                err_cnt <= err_cnt + X"0001";
            elsif (signed(m_axis_tdata(23 downto 0)) > signed(value2_up_out)) then
                report "Right output does not match, expected " & integer'image(to_integer(signed(value2_std_logic_24_bit_out))) 
                & " got " & integer'image(to_integer(signed(m_axis_tdata(23 downto 0)))) severity warning;
                err_cnt <= err_cnt + X"0001";
                end if;
        end if;
    end if;

end process;
\end{lstlisting}

Due to rounding methods some of the VHDL simulation output values weren't coherent with the Python values, so a tolerance is implemented (in this case a tolerance of 2 was selected).

\begin{figure}[!h]
    \centering
    \includegraphics[width=1.0\textwidth]{images/FIR_out.pdf}
    \caption{FIR filter response with a generic sine waves.}
    \label{fig:FIR_response}
\end{figure}

\begin{figure}[!h]
    \centering
    \includegraphics[width=1.0\textwidth]{images/fft_plot_log.pdf}
    \caption{Frequency analysis of the FIR filter with the given configuration.}
    \label{fig:FIR_freq_analysis}
\end{figure}

\subsection{Real-world validation}



\section{Conclusion}
In this assignment we present two different architectures of a FIR filter implemented in FPGA hardware. We exploit I$^2$S protocol and modules provided by Digilent to exchange and sample the audio stream.

\begin{thebibliography}{99}
    
        \bibitem{Digilent} 
        Digilent website, 
        \url{https://digilent.com/reference/pmod/pmodi2s2/start?redirect=1}
        
        \bibitem{Fir_github}
        Github repository, 
        \url{https://github.com/Gabriele-bot/MAPD_LAB/tree/main/FIR_project}
        
    \end{thebibliography}

\end{document}
